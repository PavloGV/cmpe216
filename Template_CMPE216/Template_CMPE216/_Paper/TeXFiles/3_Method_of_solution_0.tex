\section{Method of solution}
\label{method}
%====================
\indent This part of the proposal will delve into the math that will be used to further detail the design of the robot. Additionally, a similulation will be discussed here. The simulatin will take into account air resistance, gravity, mass of the robot, its structure, the payload, and other details. The simulation will be used to determine a simple and realizable geometric structure for the robot. \\

\indent The simulation will look at a number of physical aspects of the robot, but will focus on two main things. The first is energy usage and finding a similar method to store comparable energy density in a small robot. The discharge rate of the energy used to propel the robot upward will be simulated. The second is the flight of the robot. The wing design, air resistance, and lift will be simulated.\\

\indent Once the simulation is completed, a stretch goal for this project will be fabricate and construct the robot based on the simulation. A simple 8-bit microcontroller will be used to control the robot, its wings, the jumping mechanism, and possible safety features.\\

