%% This is file `elsarticle-template-1a-num.tex',
%%
%% Copyright 2009 Elsevier Ltd
%%
%% This file is part of the 'Elsarticle Bundle'.
%% ---------------------------------------------
%%
%% It may be distributed under the conditions of the LaTeX Project Public
%% License, either version 1.2 of this license or (at your option) any
%% later version.  The latest version of this license is in
%%    http://www.latex-project.org/lppl.txt
%% and version 1.2 or later is part of all distributions of LaTeX
%% version 1999/12/01 or later.
%%
%% The list of all files belonging to the 'Elsarticle Bundle' is
%% given in the file `manifest.txt'.
%%
%% Template article for Elsevier's document class `elsarticle'
%% with numbered style bibliographic references
%%
%% $Id: elsarticle-template-1a-num.tex 151 2009-10-08 05:18:25Z rishi $
%% $URL: http://lenova.river-valley.com/svn/elsbst/trunk/elsarticle-template-1a-num.tex $



%%
%\documentclass[twocolumn,preprint,authoryear,12pt] {elsarticle}
\documentclass[preprint,authoryear,12pt] {elsarticle}

%\documentclass[preprint,12pt]{elsarticle}
%\smartqed  % flush right qed marks, e.g. at end of proof
\usepackage{graphicx}
\usepackage{subfigure}
\usepackage{amsmath,amssymb,amstext}
\usepackage{soul}
\usepackage{enumerate}
\usepackage{array}
\usepackage{verbatim}
\usepackage{float}
\usepackage{epsf,exscale}
\usepackage[usenames,dvipsnames]{color}
\usepackage{booktabs}
\usepackage{multicol}
\usepackage{tikz}
\usepackage{psfrag}
\usetikzlibrary{shapes,arrows}
\usepackage{mathptmx}      % use Times fonts if available on your TeX system
%\usepackage{amssymb}
%\usepackage{color,graphicx,psfrag,amssymb,amsfonts,amsmath,subfigure,enumerate,float}
\usepackage{ulem}
%\renewcommand{\baselinestretch}{2}
\renewcommand{\baselinestretch}{1}
\usepackage{fullpage}
\renewcommand{\d}{\text{d}}
\renewcommand{\Re}{\text{Re}}
\renewcommand{\Im}{\text{Im}}

\newenvironment{packed_enum}{
\begin{enumerate}
  \setlength{\itemsep}{1pt}
  \setlength{\parskip}{0pt}
  \setlength{\parsep}{0pt}
}{\end{enumerate}}

\newenvironment{packed_item}{
\begin{itemize}
  \setlength{\itemsep}{1pt}
  \setlength{\parskip}{0pt}
  \setlength{\parsep}{0pt}
}{\end{itemize}}

\newenvironment{packed_description}{
\begin{description}
  \setlength{\itemsep}{1pt}
  \setlength{\parskip}{0pt}
  \setlength{\parsep}{0pt}
}{\end{description}}

%===================================
%======= FRONT MATTER =================
%===================================

\journal{UCSC CMPE 216 Bio-Inspired Locomotion (Fall 2018)}

\begin{document}

\begin{frontmatter}

%\title{On the solution of the contact problem involving inhomogeneously elastic bonded layered solids: an analytic approach} 
\title{Hummingbird-Inspired High-Speed Deceleration and Flea-Inspired Vertical Take-Off} 

\author{Pavlo Vlastos \footnote{\textit{pvlastos@ucsc.edu}}}

\address{Baskin School of Engineering, UC Santa Cruz}

\begin{abstract} 
A proposal for a bio-inspired robot and simulation are discussed. A flea-like jumping mechanism is proposed along with a hummingbird-inspired wing system for accurate navigation and deceleration. The primary goal of the robot is to deliver a fragile payload, an egg, to the a desired height and location. This proposal shows initial possible design choices.\\
\end{abstract}

\begin{keyword}
Flight \sep Bio-inspired \sep Hummingbirds \sep Fleas
\end{keyword}

\end{frontmatter}

%===================================
%======= PAPER =======================
%===================================

\section{Introduction}
\label{intro}
%===============

The ability of certain species (insects, geckos etc.) to climb or cling to surfaces has been observed for many years. However, the underlying physics, leading to this ability, start being understood only recently. Time, habitat and evolution have yielded in different solutions for adhesive ability of the various species, where the complexity of the mechanism seems to be directly proportional to the size of the creature. Since the heaviest of these species is the family of gecko lizards, most research has been directed towards understanding their adhesive ability. It has been shown that the principal mechanism responsible is a very skillful use of ''directional friction'' between nano-scale structures occurring on the gecko toe skin and the environment. Recent analytical work has shown that the spatula-like structure at the end of the natural gecko nanofibers is a critical element in the gecko adhesive system. This provides the ability to get intimate contact with rough surfaces, significantly increases the effective contact area and therefore the adhesive properties of the gecko foot. \cite{ma1995contact, kosaka1992fast, stone2004application}. 

When a GSAs nanofibrillar array is adhering to a surface, fibers undergo non-linear deformation of both the stems and the tip. The polymers used for manufacturing the fibers display plastic creep even at relatively low strain rates and stresses below plastic yield. Therefore, a suitable numerical solution, which predicts the optimal fiber geometry, must consider not only the initial shape of the fiber, but also the fiber progressive deformation (local and global) and the influence this has on the local mechanical properties (elastic, viscoelastic, strain hardening/softening and plastic flow). Modeling all these analytically is impossible and using traditional modeling techniques (e.g. finite element methods) could be prohibitive due to the nonlinear nature of the phenomena (e.g. viscoelastic behavior and creep) and the continuously changing boundary conditions (e.g. material relocation). A suitable modeling technique is the Smoothed-Particle Hydrodynamics (SPH). This is a mesh free Lagrange method that considers a particle array, which interact based on predefined local potentials, accounting for the continuously evolving shape (e.g. creep or plastic flow) and material properties (e.g. strain softening) much more easily than the FEM does 

The current paper is a parametric study of realistic (modeling existing GSAs) arrays of soft compressible viscoelastic polymer fibers. The deformation, adhesion and shear/friction resistance will be predicted using SPH as a function of the polymer constitutive model (describing different polymers) fiber geometry (different manufacturing techniques and/or shapes) and loading conditions (adhesion on different media and/or loading cycles). 

\subsection{FEM vs. Meshfree}
\label{FEM_vs_SPH}
The Finite Element Method (FEM) represents the traditional approach of predicting the stresses, strains and deformations in an elastic solid. 

\begin{packed_enum}
\item Mesh distortion. This could either terminate the simulation of result in significant errors. The solution: re-meshing.
\item High gradient or distinct local character: very fine mesh = computationally expensive. The solution: adaptive FEM.
\item Cracks, fragmentations, explosions, impact/penetration: the problems are not only with re- meshing, but the also with mapping the state variables (e.g. thermodynamics variables)
\item Failure mechanics: in FEM simulated material material disintegration is in fact disintegration of the FEM subdivision. therefore, the simulated disintegration pattern was integrated in the model before the simulation ever started (e.g mesh sensitivity in crack growth simulations). 
\end{packed_enum}

\subsection {SPH}
\label{SPH}
\begin{packed_item}
\item Meshfree methods are a natural extension of the FEM. However:
It cold be computationally efficient and physically more accurate to discretize a continuum by only a set of nodal points (particles) without adding mesh constraints.
\item Smoothed particle hydrodynamics (SPH) is a gridless Lagrangian particle method which replaces the continuum equations of fluid dynamics with particle equations.
\item The interactions between the particles are determined by interpolation from information at the SPH particles.
\item When SPH is applied to solids the SPH particles mimic the behavior of the atoms: If the solid is compressed the atoms repel each other. If it is stretched the atoms attract each other and oppose the \ref{Fibers_Bending} stretch \citealp{Tim_Good}.
\end{packed_item}




\section{Formulation of the problem}
\label{Formulation}
%==========================
SPH is a mesh free Lagrange method that starts with the creation of a particle array and is governed by momentum equations, which determine change in density and viscosity. The kernel function $W_{ab}$ is a function of the distance $r_{ab}$ between particle $a$ and $b$.

The interaction between the particle $a$ and the neighboring particles is considered through a smoothening kernel function. In the current paper we used cubic spline kernel .... (see equation \ref{kernel}), where $h$ is the... and $q=r_{ab}/h$ represents the normalized distance between the particle $a$ and $b$. Figure \ref{kernel_figure} show a schematic representation of the kernel function.....


\begin{figure}[h]
\centering
\subfigure[We should redraw this]{\includegraphics[width=0.5\linewidth]{./_Figures/Kernel.jpg} } 
\subfigure[another generic picture explaining the formulation ]{\includegraphics[width=0.2\linewidth]{./_Figures/slug.jpg} } 
\caption{array of soft compressible viscoelastic polymer fibers compressed by an elastic wall – the colors represent non-dimensional particle density, where ”1” is the density of the undeformed fibers}
\label{kernel_figure}
\end{figure}


\begin{align}
\label{kernel}
W(r,h)=
\begin{cases}
\frac{10}{7 \pi h^2} \left( 1 - \frac{3}{2}q^2 + \frac{3}{4}q^3 \right), \qquad q<1,\\
\frac{5}{14 \pi h^2} \left( 2-q \right)^3, \qquad \qquad 1<q<2,\\
0, \qquad \qquad \qquad \qquad \qquad 2<q
\end{cases}
\end{align}

\begin{subequations}
\begin{align}
\frac{\partial\rho_a}{\partial t}&=\sum_{b=1}^N m_b\left( \vec v_b- \vec v_a\right) \bigtriangledown_a W_{ab},
\label{eq1}\\
\frac{\vec v_a}{\partial t}&=-\sum_{b=1}^N m_b\left( \frac{p_a}{\rho_a^2}+\frac{p_b}{\rho_b^2}\right) \bigtriangledown_a W_{ab},
\label{eq2}\\
\frac{\vec x_a}{\partial t}&=\vec v_a,
\label{eq3}
\end{align}
\end{subequations}

where $N$ is the total number of particles at distance $|\vec x_a-\vec x_b|<h$, $\rho_i$ is the density, $m_i$ the mass and $\vec v_i$ the velocity of a particle $i \in \{a,b\}$. $\bigtriangledown W$ is the gradient of a smoothing kernel function, $p$ is the pressure applied on a particle, {\color{red} which is obtained from the energy equation. If the SPH method is used to model a fluid, this is given by Tait equation. If, however, a solid is modeled, the pressure is predicted by teh material constitutive model.}

Therefore, for an elastic solid, equation \ref{eq2} could be rewritten as:

\begin{equation}
\frac{\partial v_a^i}{\partial t}=\sum_{b=1}^N m_b\left( \frac{\sigma_a^{ij}}{\rho_a^2}+\frac{\sigma_b^{ij}}{\rho_b^2}+\Pi_{ab}\delta^{ij}\right) \frac{\partial W_{ab}}{\partial x_a^i} +g^i,
\label{eq2_2}
\end{equation}

where $\sigma^{ij}$ is teh stress tensor, the term $\Pi_{ab}\delta^{ij}$ represents teh shear bulk viscosity and $g^i$ is the $i^{\text{th}}$ component of teh body force per unit mass.


%\begin{align}
%\mu(y)=
%\begin{cases}
%\mu_1e^{\alpha y}, \qquad -h\leq y\leq 0,\\
%\mu_0, \qquad -\infty<y<-h
%\end{cases}
%\end{align}

%\begin{subequations}
%\begin{align}
%\frac{\partial\sigma_{xx}}{\partial x}+\frac{\partial\sigma_{xy}}{\partial y}&=0,\label{equil_1}\\
%\frac{\partial\sigma_{xy}}{\partial x}+\frac{\partial\sigma_{yy}}{\partial y}&=0.\label{equil_2}
%\end{align}
%\end{subequations}

%\begin{align}
%\sigma_{yy}(x,0)&=P(x),\notag\\
%\sigma_{xy}(x,0)&=0,
%\end{align}


 	
\section{Method of solution}
\label{method}
%====================
\indent This part of the proposal will delve into the math that will be used to further detail the design of the robot. Additionally, a similulation will be discussed here. The simulatin will take into account air resistance, gravity, mass of the robot, its structure, the payload, and other details. The simulation will be used to determine a simple and realizable geometric structure for the robot. \\

\indent The simulation will look at a number of physical aspects of the robot, but will focus on two main things. The first is energy usage and finding a similar method to store comparable energy density in a small robot. The discharge rate of the energy used to propel the robot upward will be simulated. The second is the flight of the robot. The wing design, air resistance, and lift will be simulated.\\

\indent Once the simulation is completed, a stretch goal for this project will be fabricate and construct the robot based on the simulation. A simple 8-bit microcontroller will be used to control the robot, its wings, the jumping mechanism, and possible safety features.\\

 		
\section{Implementation} 
\label{implementation}
%===============

The algorithm, the numerical solution etc.

%===================== chart Mircea ==========
\begin{figure}[H]
\begin{center}
%\begin{comment}
% Define block styles
\tikzstyle{decision} = [aspect=3, diamond, draw, fill=white!0, text width=6em, text centered, node distance=3cm, inner sep=0pt, minimum height=4em]
\tikzstyle{block} = [rectangle, draw, fill=white!0,text width=8em, text centered, rounded corners, minimum height=2em]
\tikzstyle{block_1} = [rectangle, draw, fill=white!0,text width=19em, text centered, rounded corners, minimum height=2em]
\tikzstyle{line} = [draw, -latex']
    
\begin{tikzpicture}[node distance = 2cm, auto]
% Place nodes
    	\node [block] (P) {Input Pressure $P(x)$};
    	\node [decision, below of=P, node distance=1.8cm] (decide_1) {is Pressure \\ Continuous ?};
    	\node [block, below left of=decide_1, node distance=3.2cm] (No) {Fit linear Spline to $P(x)$};
    	\node [block, below of=No, node distance=2cm] (No_1) {Compute Fourier coefficients $P_n$ using (\ref{discrete_Fourier_coefficients})};
    	\node [block, below right of=decide_1, node distance=3.2cm] (Yes) {Compute Fourier coefficients $P_n$ exactly using (\ref{cont_Fourier_coefficients})};
	\node [block_1, below of=decide_1, node distance=6.2cm] (Lambda) {Compute constants in (\ref{lambda}), (\ref{gamma}), (\ref{delta}), (\ref{tau}) \\ $\lambda_{j,n}, \gamma_{j,n}, \delta_n \tau_{j,n}; j=1..4$};
	\node [block_1, below of=Lambda, node distance=1.8cm] (Mat) {Compute Matrices in (\ref{matrices_1}), (\ref{matrices_2}) \\ $\mathcal{K}_{j,n}, M_{j,n}, N_{j,n}, Z_{j,n}, W_nj=1..4$};
    	\node [block_1, below of=Mat, node distance=1.8cm](Coef){Calculate Coefficients in (\ref{A_coeff_odd}), (\ref{A_coeff_even}), (\ref{C_coeff}) \\ $A_n^{(1..4)}, C_n^{(1..2)}$};
    \node [block_1, below of=Coef, node distance=1.6cm](uv){Construct $u, v$ using (\ref{final_coat_solution_orig}),  (\ref{final_substrate_solution_orig})};
    \node [block_1, below of=uv, node distance=1.7cm](tau){Determine $\sigma_{xx}, \sigma_{yy}, \sigma_{xy}$ using (\ref{sigma_dimensionless}) and $\tau_1$  using (\ref{tau_1}) };
    
% Draw lines
    \path [line] (P) -- (decide_1);
    \path [line] (decide_1) -- node [above left] {No} (No);
    \path [line] (No) -- (No_1);
    \path [line] (decide_1) -- node [above right]{Yes} (Yes);
    \path [line] (No_1) -- (Lambda);
    \path [line] (Yes) -- (Lambda);
    \path [line] (Lambda) -- (Mat);
    \path [line] (Mat) -- (Coef);
    \path [line] (Coef) -- (uv);
    \path [line] (uv) -- (tau);
 
\end{tikzpicture}
%\end{comment}

%\includegraphics[width=0.4\textwidth]{./Figures/Chart.pdf}
\caption{Example of a chart}
\label{fig:chart}
\end{center}
\end{figure} 
%===================== End chart ========== 		
\section{Numerical results and discussions}
\label{results}
%============

text text

\begin{figure}[H]
\centering
\subfigure[undeformed array of fibers]{\includegraphics[width=0.32\linewidth]{./_Figures/Fibers_a.jpg} } 
\subfigure[deformed array of fibers]{\includegraphics[width=0.32\linewidth]{./_Figures/Fibers_b.jpg} } 
\caption{array of soft compressible viscoelastic polymer fibers compressed by an elastic wall – the colors represent non-dimensional particle density, where ”1” is the density of the undeformed fibers}
\label{Fibers_Bending}.
\end{figure}
 	

% ADD THIS AT THE END ____
%\section*{Acknowledgement}
% ADD THIS AT THE END ____

% \section*{References}
\bibliography{./TeXFiles/Bib_Stewart}
\bibliographystyle{./model3-num-names.bst}

\end{document}

