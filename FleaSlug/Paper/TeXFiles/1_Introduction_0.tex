\section{Introduction}
\label{intro}
%===============

\indent Some robots require specific infrastructure and where as some do not. A rumba for example requires flat surfaces to traverse, where as DARPA's Big Dog can navigate the uneven surface of a grassy hill. Robots that do not require specific environmental geometry can be deployed in countries that have limited infrastructure to deliver important supplies, conduct search and rescue, and study hazardous environments. In the case of disaster relief, supply delivery via all-terrain, long-distance autonomous vehicles can be critical to developing nations.\\

\indent This paper outlines a proposal for a high-velocity aerial robot drawing
inspiration from hummingbirds and fleas. The robot will use vertical take-off, requiring no runway. Additional bio-inspiration is drawn from the manner in which humming-birds rapidly decelerate. When hummingbirds are in mating season, the males attract females executing a looping \textit{display dive} reaching speeds of up to approximately 27 meters per second, \cite{bennet-clark_jump_nodate}. The system incorporates an initial acceleration mechanism to propel the robot to a high altitude. Once the apex of the trajectory is reached, hummingbird-inspired wings and tail feathers will be deployed to mimic the deceleration.\\

\indent The vertical take-off mechanism is inspired from the jumping mechanism of a flea. Fleas are capable of jumping with an acceleration of 102 g-forces, \cite[p.~62]{bennet-clark_jump_nodate}. This is a tremendous acceleration that is not easily reverse engineered. This proposal will explore a design for a jumping mechanism based on scaling the acceleration capable of fleas, but with a larger mass. A comparison of energy density relative to size of a payload will be made. A simple free-body diagram is outlined in \textbf{section} \ref{Formulation}. The forward and inverse kinematics are also discussed in \textbf{section} \ref{Formulation}.\\

\indent  

\indent The basic design of the robot will be broken up in to two main sections: \textbf{1)} the initial acceleration mechanism and \textbf{2)} the deceleration mechanism. A simulation will first be made, and as a stretch-goal, a prototype will be assembled.\\