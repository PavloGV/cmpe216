\section{Introduction}
\label{intro}
%===============

\indent Some robots require specific infrastructure and where as some do not. A rumba for example requires flat surfaces to traverse, where as DARPA's Big Dog can navigate the uneven surface of a grassy hill. Robots that do not require specific environmental geometry can be deployed in countries that have limited infrastructure to deliver important supplies, conduct search and rescue, and study hazardous environments. In the case of disaster relief, supply delivery via all-terrain, long-distance autonomous vehicles can be critical to developing nations.\\

\indent This paper outlines a proposal for a high-velocity aerial robot drawing
inspiration from hummingbirds and fleas. The robot will use vertical take-off, requiring no runway. Additional bio-inspiration is drawn from the manner in which humming-birds rapidly decelerate. When hummingbirds are in mating season, the males attract females executing a looping \textit{display dive} reaching speeds of up to approximately 27 meters per second, \cite{bennet-clark_jump_nodate}. The system incorporates an initial acceleration mechanism to propel the robot to a high altitude. Once the apex of the trajectory is reached, hummingbird-inspired wings and tail feathers will be deployed to mimic the deceleration.\\

\indent The vertical take-off mechanism is inspired from the jumping mechanism of a flea. Fleas are capable of jumping with an acceleration of 102 g-forces, \cite[p.~62]{bennet-clark_jump_nodate}. This is a tremendous acceleration that is not easily reverse engineered. This proposal will explore a design for a jumping mechanism based on scaling the acceleration capable of fleas, but with a larger mass. A comparison of energy density relative to size of a payload will be made. A simple free-body diagram is outlined in \textbf{section} \ref{Formulation}. The forward and inverse kinematics are also discussed in \textbf{section} \ref{Formulation}.\\

\indent One example of a jumping robot is a 7 gram robot capable of jumping over 1 meter in height. The approach to designing the 7 gram robot structure was largely based off of evaluating the kinetic energy required for the robot to reach a certain height. The robot was designed with the idea of carrying a payload, so a 'cost of payload' was used to further estimate the energy efficiency. An analysis of the energy stored in the spring and jumping height was conducted to see the effects of carrying a 3 gram payload and without any payload. One of the features of the design was an adjustable takeoff angle. The design was theoretically capable of jumping 108 times for a height difference of up to 148 meters, \cite{kovac_miniature_2008}. \\

\indent Another jumping robot was designed with an emphasis on power modulation of a series elastic actuator. To achieve power modulation in a muscle, the instantaneous power of said muscle complex must exceed the most energy that could be output by the muscle by itself. The muscle transfers energy to parallel-elastic structures and then that energy is released at energy levels that could not otherwise be realized by the muscle without the parallel-elastic structures. Essentially, this particular robot used a temporary, physical energy storage medium to amplify its jumping potential. Using this elastic structure to temporary store energy is a weight efficient method for achieving higher jump heights. Part of the inspiration for the robot design was the galago, which also exhibits power modulation, and posses the highest vertical jumping agility. The model for the design was a simple mass, spring, and damper. The spring and damper were in series and could push or pull the leg of the robot. The robot was found to be able to jump higher to 78\% of the galago's jumping agility, where as previous robots were only capable of up to 55\%, \cite{haldane_robotic_2016}. As with the previous robot example, this galago-inspired robot was also designed with respect to a weight and jumping-energy usage relation. Its efficiency however was greater than the previous.\\

\indent A third design emphasized stability over jumping height. The Gearless Omni-direction Acceleration-vectoring Topology (GOAT) used a 3-DoF design. It is capable of jumping in any direction on a plane. It uses high fidelity proprioceptive force control. While its maximum jump height is 82 centimeters, it can land on the same leg used for jumping to jump repeatedly. This contrasts with galago-inspired robot, as it was only able to jump a few times in succession. The GOAT is also capable of running-jumping trajectories, albeit mounted to a test rig. To further contrast, instead of using series-elastic actuators (SEAs), like the galago-inspired robot, the GOAT uses virtual model control. This uses motors to emulate the dynamics of series-elastic components and various other mechanical components. The use of direct-drive and quasi-direct drive actuators further assist in avoiding possible bandwidth limitations of series-elastic actuators, \cite{kalouche_goat:_2017}. The GOAT is more geared toward smaller jumps and specifically running-jumping gates of motion in any direction on a plane. Jump efficiency is not compared to something such as a galago.\\

\indent The basic design of the robot will be broken up in to two main sections: \textbf{1)} the initial acceleration mechanism and \textbf{2)} the deceleration mechanism. A simulation will first be made, and as a stretch-goal, a prototype will be assembled.\\